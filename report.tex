% Options for packages loaded elsewhere
\PassOptionsToPackage{unicode}{hyperref}
\PassOptionsToPackage{hyphens}{url}
\PassOptionsToPackage{dvipsnames,svgnames,x11names}{xcolor}
%
\documentclass[
  letterpaper,
  DIV=11,
  numbers=noendperiod]{scrartcl}

\usepackage{amsmath,amssymb}
\usepackage{iftex}
\ifPDFTeX
  \usepackage[T1]{fontenc}
  \usepackage[utf8]{inputenc}
  \usepackage{textcomp} % provide euro and other symbols
\else % if luatex or xetex
  \usepackage{unicode-math}
  \defaultfontfeatures{Scale=MatchLowercase}
  \defaultfontfeatures[\rmfamily]{Ligatures=TeX,Scale=1}
\fi
\usepackage{lmodern}
\ifPDFTeX\else  
    % xetex/luatex font selection
\fi
% Use upquote if available, for straight quotes in verbatim environments
\IfFileExists{upquote.sty}{\usepackage{upquote}}{}
\IfFileExists{microtype.sty}{% use microtype if available
  \usepackage[]{microtype}
  \UseMicrotypeSet[protrusion]{basicmath} % disable protrusion for tt fonts
}{}
\makeatletter
\@ifundefined{KOMAClassName}{% if non-KOMA class
  \IfFileExists{parskip.sty}{%
    \usepackage{parskip}
  }{% else
    \setlength{\parindent}{0pt}
    \setlength{\parskip}{6pt plus 2pt minus 1pt}}
}{% if KOMA class
  \KOMAoptions{parskip=half}}
\makeatother
\usepackage{xcolor}
\setlength{\emergencystretch}{3em} % prevent overfull lines
\setcounter{secnumdepth}{5}
% Make \paragraph and \subparagraph free-standing
\makeatletter
\ifx\paragraph\undefined\else
  \let\oldparagraph\paragraph
  \renewcommand{\paragraph}{
    \@ifstar
      \xxxParagraphStar
      \xxxParagraphNoStar
  }
  \newcommand{\xxxParagraphStar}[1]{\oldparagraph*{#1}\mbox{}}
  \newcommand{\xxxParagraphNoStar}[1]{\oldparagraph{#1}\mbox{}}
\fi
\ifx\subparagraph\undefined\else
  \let\oldsubparagraph\subparagraph
  \renewcommand{\subparagraph}{
    \@ifstar
      \xxxSubParagraphStar
      \xxxSubParagraphNoStar
  }
  \newcommand{\xxxSubParagraphStar}[1]{\oldsubparagraph*{#1}\mbox{}}
  \newcommand{\xxxSubParagraphNoStar}[1]{\oldsubparagraph{#1}\mbox{}}
\fi
\makeatother


\providecommand{\tightlist}{%
  \setlength{\itemsep}{0pt}\setlength{\parskip}{0pt}}\usepackage{longtable,booktabs,array}
\usepackage{calc} % for calculating minipage widths
% Correct order of tables after \paragraph or \subparagraph
\usepackage{etoolbox}
\makeatletter
\patchcmd\longtable{\par}{\if@noskipsec\mbox{}\fi\par}{}{}
\makeatother
% Allow footnotes in longtable head/foot
\IfFileExists{footnotehyper.sty}{\usepackage{footnotehyper}}{\usepackage{footnote}}
\makesavenoteenv{longtable}
\usepackage{graphicx}
\makeatletter
\def\maxwidth{\ifdim\Gin@nat@width>\linewidth\linewidth\else\Gin@nat@width\fi}
\def\maxheight{\ifdim\Gin@nat@height>\textheight\textheight\else\Gin@nat@height\fi}
\makeatother
% Scale images if necessary, so that they will not overflow the page
% margins by default, and it is still possible to overwrite the defaults
% using explicit options in \includegraphics[width, height, ...]{}
\setkeys{Gin}{width=\maxwidth,height=\maxheight,keepaspectratio}
% Set default figure placement to htbp
\makeatletter
\def\fps@figure{htbp}
\makeatother

\KOMAoption{captions}{tableheading}
\makeatletter
\@ifpackageloaded{caption}{}{\usepackage{caption}}
\AtBeginDocument{%
\ifdefined\contentsname
  \renewcommand*\contentsname{Table of contents}
\else
  \newcommand\contentsname{Table of contents}
\fi
\ifdefined\listfigurename
  \renewcommand*\listfigurename{List of Figures}
\else
  \newcommand\listfigurename{List of Figures}
\fi
\ifdefined\listtablename
  \renewcommand*\listtablename{List of Tables}
\else
  \newcommand\listtablename{List of Tables}
\fi
\ifdefined\figurename
  \renewcommand*\figurename{Figure}
\else
  \newcommand\figurename{Figure}
\fi
\ifdefined\tablename
  \renewcommand*\tablename{Table}
\else
  \newcommand\tablename{Table}
\fi
}
\@ifpackageloaded{float}{}{\usepackage{float}}
\floatstyle{ruled}
\@ifundefined{c@chapter}{\newfloat{codelisting}{h}{lop}}{\newfloat{codelisting}{h}{lop}[chapter]}
\floatname{codelisting}{Listing}
\newcommand*\listoflistings{\listof{codelisting}{List of Listings}}
\makeatother
\makeatletter
\makeatother
\makeatletter
\@ifpackageloaded{caption}{}{\usepackage{caption}}
\@ifpackageloaded{subcaption}{}{\usepackage{subcaption}}
\makeatother

\ifLuaTeX
  \usepackage{selnolig}  % disable illegal ligatures
\fi
\usepackage{bookmark}

\IfFileExists{xurl.sty}{\usepackage{xurl}}{} % add URL line breaks if available
\urlstyle{same} % disable monospaced font for URLs
\hypersetup{
  pdftitle={Methodology for Comparing Citation Database Coverage of Dataset Usage},
  colorlinks=true,
  linkcolor={blue},
  filecolor={Maroon},
  citecolor={Blue},
  urlcolor={Blue},
  pdfcreator={LaTeX via pandoc}}


\title{Methodology for Comparing Citation Database Coverage of Dataset
Usage}
\usepackage{etoolbox}
\makeatletter
\providecommand{\subtitle}[1]{% add subtitle to \maketitle
  \apptocmd{\@title}{\par {\large #1 \par}}{}{}
}
\makeatother
\subtitle{Findings}
\author{}
\date{2025-04-26}

\begin{document}
\maketitle

\renewcommand*\contentsname{Table of contents}
{
\hypersetup{linkcolor=}
\setcounter{tocdepth}{3}
\tableofcontents
}

\section{Overview}\label{overview}

\href{report.pdf}{Download PDF Version}

This report compares the differences between the Scopus and OpenAlex
citation databases in their tracking of dataset mentions. The Census of
Agriculture, produced by the USDA National Agricultural Statistical
Services (NASS), provides information on U.S. farming operations,
including production practices and land use. This dataset is used as a
case study in this report to systematically compare the coverage,
overlap, and differences in publications indexed by the two citation
databases. The Census of Agriculture (also referred to as the ``Ag
Census'') is selected for its frequent use in agricultural and economic
research, making it an ideal dataset for assessing differences in
publication coverage between Scopus and OpenAlex.

\section{Data Collection}\label{data-collection}

To compare coverage across the two citation databases, publications that
mention the Ag Census must first be identified. The methods used to
identify dataset mentions in Scopus and OpenAlex are described below.

\subsection{Scopus Approach}\label{scopus-approach}

The first citation database used is Scopus, a publication catalog
managed by Elsevier. Ideally, direct Scopus API access would have been
used to query full publication text for mentions of the Census of
Agriculture. However, the project did not have access to the Scopus API.
Only Elsevier, serving as a project partner, was able to execute queries
within the Scopus environment. Consequently, the dataset mention search
relied on outputs provided by Elsevier rather than independent querying.

Because of these constraints, a seed corpus approach was applied. First,
Elsevier matched the names and aliases of selected datasets, including
the Census of Agriculture, against full-text records available through
ScienceDirect and reference sections of Scopus publications published
between 2017 and 2023. This initial step identified journals, authors,
and topics most likely to reference the Ag Census. A targeted search
corpus was then constructed, narrowing the scope to approximately 1.45
million publications.

Several methods were used to identify mentions of USDA datasets in
Scopus publications. First, a reference search was conducted, using
exact-text matching across publication reference lists to capture formal
citations of datasets. Second, full-text searches were performed using
machine learning models applied to publication bodies, identifying less
formal mentions of datasets. Third, machine learning routines developed
through the 2021 Kaggle competition were applied to the full-text corpus
to improve detection of dataset mentions, including instances where
references were indirect or less structured. Details about the three
machine learning models used are available
\href{https://github.com/democratizingdata/democratizingdata-ml-algorithms/blob/main/README.md}{here}.

Because direct access to full publication text was not available,
Elsevier shared only the extracted snippets and limited metadata. Manual
validation, aided by the use of keyword flags (e.g., ``USDA,''
``NASS''), confirmed whether identified mentions accurately referred to
the Census of Agriculture. To manage validation costs, only publications
with at least one U.S.-based author were reviewed.

Full documentation of the Scopus search routine, including query
construction and extraction procedures, is available at the project's
\href{https://laurenchenarides.github.io/compare_scopus_openalex_report/workflow/step02_01/extract_dataset_mentions.html}{report
website}.

\subsection{OpenAlex Approach}\label{openalex-approach}

The second citation database used is OpenAlex, an open catalog of
scholarly publications. OpenAlex offers public access to metadata and,
when available, full-text content for open-access publications through
its
\href{https://docs.openalex.org/how-to-use-the-api/api-overview}{API}.
Unlike Scopus, which provides controlled access to licensed content,
OpenAlex indexes only publications that are openly available or for
which open metadata has been provided by publishers.

For OpenAlex, two approaches were used to identify publications
referencing the Census of Agriculture. The first approach relied on a
full-text search across OpenAlex publication records. The second
approach applied a seed corpus methodology, similar to the strategy used
for Scopus, to address limitations observed in the initial full-text
search.

\subsubsection{Full-Text Search
Approach}\label{full-text-search-approach}

The methodology for collecting mentions of USDA datasets in OpenAlex
relied on constructing search queries that combined dataset ``aliases''
and associated ``flag terms'' within the text of scholarly works.
Dataset aliases represented alternative ways researchers refer to a
dataset, such as variations on the Census of Agriculture's official
name. Flag terms represented the institutions or agencies responsible
for maintaining the dataset. The combination of dataset alias and flag
terms ensured that retrieved publications made an explicit connection to
the correct data source. A mention was recorded only if at least one
alias and one flag term appeared in the same publication, thereby
increasing the likelihood of capturing genuine dataset references rather
than incidental matches to individual
words.\footnote{Initial drafts of the query incorrectly included terms like "NASS" and "USDA" in the alias list. This was corrected to ensure that aliases strictly referred to dataset names, and flag terms referred to organizations.}

To implement these searches efficiently, the OpenAlex API was accessed
using the \texttt{pyalex} Python
package.\footnote{`Pyalex` is an open-source library designed to facilitate interaction with the OpenAlex API; see [https://help.openalex.org/hc/en-us/articles/27086501974551-Projects-Using-OpenAlex](https://help.openalex.org/hc/en-us/articles/27086501974551-Projects-Using-OpenAlex) for more information. The package manages request formatting and automates compliance with OpenAlex's "polite pool" rate limits, which restrict the number of requests per minute and impose backoff delays. Pyalex introduced automatic pauses between requests, with a default `retry_backoff_factor` of 100 milliseconds, to ensure stable and continuous retrieval. This setup enabled systematic querying while adhering to OpenAlex's usage policies.}

Search queries were constructed based on OpenAlex's public API
documentation, using both the
\href{https://docs.openalex.org/api-entities/works/filter-works}{``Filter
Works''} and
\href{https://docs.openalex.org/api-entities/works/search-works}{``Search
Works''} endpoints. Filtering parameters were applied to restrict
results to English-language publications, published after 2017,
classified as articles or reviews, and available through open-access
sources.

Boolean logic was used to define the text search structure. For the
Census of Agriculture, the query grouped several dataset aliases,
including ``Census of Agriculture,'' ``USDA Census of Agriculture,''
``Agricultural Census,'' and ``USDA Census.'' These aliases were
combined using an \texttt{OR} operator. Separately, flag terms including
``USDA,'' ``U.S. Department of Agriculture,'' ``United States Department
of Agriculture,'' ``NASS,'' and ``National Agricultural Statistics
Service'' were also grouped using an \texttt{OR} operator. The final
query ensured that both an alias and a flag term appeared by connecting
the two groups with an \texttt{AND} operator:

\begin{quote}
(``NASS Census of Agriculture'' OR ``Census of Agriculture'' OR ``USDA
Census of Agriculture'' OR ``Agricultural Census'' OR ``USDA Census'' OR
``AG Census'') AND (USDA OR ``US Department of Agriculture'' OR ``United
States Department of Agriculture'' OR NASS OR ``National Agricultural
Statistics Service'')
\end{quote}

This structure required that each publication mention both a recognized
variant of the Census of Agriculture name and a reference to the
institution responsible for producing it.

Publications matching the query were returned in JSON format, based on
the OpenAlex
\href{https://docs.openalex.org/api-entities/works/work-object}{``Work
object''} schema. Each record included metadata fields such as:

\begin{itemize}
  \item `display_name` (publication title)
  \item `authorships` (authors and affiliations)
  \item `host_venue.display_name` (journal)
  \item `doi` (digital object identifier)
  \item `concepts` (topics)
  \item `cited_by_count` (citation counts)
  \item `type` (publication type, e.g., "article", "book-chapter")
  \item `publication_year` (year article was publish)
  \item `language` (language, English only)
  \item `is_oa` (open access)
\end{itemize}

Although a range of publication types were retrieved---including
articles, book chapters, dissertations, preprints, and
reviews---approximately 80--85 percent were classified as articles. To
standardize the dataset for downstream analysis, results were filtered
during the search process to retain only records identified as
\texttt{type\ =\ article}. This step removed preprints and non-final
versions of works, supporting a more standardized analysis of dataset
mentions in peer-reviewed literature.

The code used to implement this querying and filtering process is
publicly available
\href{https://github.com/laurenchenarides/compare_scopus_openalex_report}{here}.

\paragraph{Limitations of Full-Text
Approach}\label{limitations-of-full-text-approach}

Although the OpenAlex API provides full-text search capabilities,
limitations in how publication content is ingested and indexed introduce
challenges for identifying dataset mentions accurately.

OpenAlex receives publication text through two primary ingestion
methods: PDF extraction and
\href{https://docs.openalex.org/api-entities/works/get-n-grams}{n-grams
delivery}. In the PDF ingestion method, OpenAlex extracts text directly
from the article PDF. However, the references section is not included in
the searchable text. References are processed separately to create
citation pointers between scholarly works, meaning that mentions of
datasets appearing only in bibliographies are not discoverable through
full-text search.

In the n-grams ingestion method, OpenAlex does not receive the full
article text. Instead, it receives a set of extracted word sequences
(n-grams) from the publisher or author. These n-grams represent
fragments of text---typically short sequences of one, two, or three
words---which are not guaranteed to preserve full continuous phrases. As
a result, complete dataset names may be broken apart or omitted,
reducing the likelihood that search queries match the intended aliases.

These ingestion and indexing limitations affect the completeness of
results when relying solely on OpenAlex full-text search. Mentions of
the Census of Agriculture and other USDA datasets that appear either
exclusively in references or are fragmented within n-grams may be
missed. To address these limitations, an alternative search strategy was
developed based on constructing a filtered seed corpus of publications
for local full-text analysis.

\subsubsection{Seed Corpus Approach}\label{seed-corpus-approach}

To address limitations in OpenAlex's full-text indexing methods, a seed
corpus approach was applied. The objective was to create a filtered set
of publications for local text search to better capture mentions of the
Census of Agriculture and related USDA datasets.

To construct the seed corpus, publications were filtered based on
several criteria:

\begin{itemize}
\item Language: English
\item Publication Year: Post-2017
\item Publication Type: Articles and reviews
\item Open Access Status: Open-access publications only
\end{itemize}

Filtering was further refined by selecting publications associated with
high-relevance topics (Table @ref(tab:top25topics) lists the top 25
topics), top publishing journals (Table @ref(tab:top25journals) lists
the top 25 journals), and U.S.-affiliated authors (Table
@ref(tab:top25authors) lists the top 25 authors).

Each table presents two key columns to help interpret the selection
process. The First Run Count refers to the number of publications linked
to each entity (whether a topic, journal, or author) based on metadata
from OpenAlex's full-text search feature. This count reflects how often
USDA datasets were mentioned within the full text of publications
associated with a particular entity. The OpenAlex Total Count represents
the total number of publications linked to that entity in the broader
OpenAlex database, without applying any filters related to dataset
mentions.

To create a more focused and manageable search corpus, we selected the
top 25 entities in each category based on their First Run Count. This
approach prioritizes journals, topics, and authors where USDA datasets
are most frequently mentioned in the full text, which we interpret as
being more representative of actual research activity involving these
datasets. It also substantially reduces the workload by limiting the
number of publications that need to be retrieved and processed.

Choosing this approach has a few important implications. First, it
likely increases the relevance of the resulting corpus by concentrating
on publications where USDA data are actively cited or discussed, rather
than simply associated with a broader research area. Second, it helps
avoid the need to download and process an unmanageable number of
PDFs---estimated at around 1.7 million if all identified entities were
included. However, this method may introduce some selection bias by
favoring entities with higher immediate visibility in the first search
pass. Some relevant but less frequently mentioned entities might be
excluded, meaning that while efficiency improves, full comprehensiveness
is slightly sacrificed. Overall, this tradeoff supports a practical
balance between depth and feasibility in building the final dataset.

\begin{longtable}[]{@{}
  >{\raggedright\arraybackslash}p{(\columnwidth - 6\tabcolsep) * \real{0.0877}}
  >{\raggedright\arraybackslash}p{(\columnwidth - 6\tabcolsep) * \real{0.5702}}
  >{\raggedright\arraybackslash}p{(\columnwidth - 6\tabcolsep) * \real{0.1491}}
  >{\raggedright\arraybackslash}p{(\columnwidth - 6\tabcolsep) * \real{0.1930}}@{}}
\caption{(\#tab:top25topics) Top 25 Topics by First Run
Count}\tabularnewline
\toprule\noalign{}
\begin{minipage}[b]{\linewidth}\raggedright
Topic ID
\end{minipage} & \begin{minipage}[b]{\linewidth}\raggedright
Topic Name
\end{minipage} & \begin{minipage}[b]{\linewidth}\raggedright
First Run Count
\end{minipage} & \begin{minipage}[b]{\linewidth}\raggedright
OpenAlex Total Count
\end{minipage} \\
\midrule\noalign{}
\endfirsthead
\toprule\noalign{}
\begin{minipage}[b]{\linewidth}\raggedright
Topic ID
\end{minipage} & \begin{minipage}[b]{\linewidth}\raggedright
Topic Name
\end{minipage} & \begin{minipage}[b]{\linewidth}\raggedright
First Run Count
\end{minipage} & \begin{minipage}[b]{\linewidth}\raggedright
OpenAlex Total Count
\end{minipage} \\
\midrule\noalign{}
\endhead
\bottomrule\noalign{}
\endlastfoot
T11610 & Impact of Food Insecurity on Health Outcomes & 549 & 78661 \\
T10010 & Global Trends in Obesity and Overweight Research & 272 &
111686 \\
T11066 & Comparative Analysis of Organic Agricultural Practices & 247 &
41275 \\
T12253 & Urban Agriculture and Community Development & 222 & 27383 \\
T10367 & Agricultural Innovation and Livelihood Diversification & 186 &
49818 \\
T11464 & Impact of Homelessness on Health and Well-being & 175 &
101019 \\
T12033 & European Agricultural Policy and Reform & 137 & 88980 \\
T10841 & Discrete Choice Models in Economics and Health Care & 126 &
66757 \\
T10596 & Maternal and Child Nutrition in Developing Countries & 116 &
118727 \\
T11898 & Impacts of Food Prices on Consumption and Poverty & 113 &
29110 \\
T11259 & Sustainable Diets and Environmental Impact & 109 & 45082 \\
T11311 & Soil and Water Nutrient Dynamics & 84 & 52847 \\
T10235 & Impact of Social Factors on Health Outcomes & 81 & 86076 \\
T10439 & Adaptation to Climate Change in Agriculture & 77 & 27311 \\
T11886 & Risk Management and Vulnerability in Agriculture & 73 &
44755 \\
T10226 & Global Analysis of Ecosystem Services and Land Use & 71 &
84104 \\
T10866 & Role of Mediterranean Diet in Health Outcomes & 70 & 76894 \\
T10969 & Optimal Operation of Water Resources Systems & 70 & 97570 \\
T10330 & Hydrological Modeling and Water Resource Management & 69 &
132216 \\
T11753 & Forest Management and Policy & 60 & 75196 \\
T12098 & Rural development and sustainability & 54 & 62114 \\
T10111 & Remote Sensing in Vegetation Monitoring and Phenology & 52 &
56452 \\
T10556 & Global Cancer Incidence and Mortality Patterns & 49 & 64063 \\
T11711 & Impacts of COVID-19 on Global Economy and Markets & 49 &
69059 \\
T12724 & Integrated Management of Water, Energy, and Food Resources & 47
& 40148 \\
\end{longtable}

\begin{longtable}[]{@{}
  >{\raggedright\arraybackslash}p{(\columnwidth - 6\tabcolsep) * \real{0.1154}}
  >{\raggedright\arraybackslash}p{(\columnwidth - 6\tabcolsep) * \real{0.5846}}
  >{\raggedright\arraybackslash}p{(\columnwidth - 6\tabcolsep) * \real{0.1308}}
  >{\raggedright\arraybackslash}p{(\columnwidth - 6\tabcolsep) * \real{0.1692}}@{}}
\caption{(\#tab:top25journals) Top 25 Journals by First Run
Count}\tabularnewline
\toprule\noalign{}
\begin{minipage}[b]{\linewidth}\raggedright
Journal ID
\end{minipage} & \begin{minipage}[b]{\linewidth}\raggedright
Journal Name
\end{minipage} & \begin{minipage}[b]{\linewidth}\raggedright
First Run Count
\end{minipage} & \begin{minipage}[b]{\linewidth}\raggedright
OpenAlex Total Count
\end{minipage} \\
\midrule\noalign{}
\endfirsthead
\toprule\noalign{}
\begin{minipage}[b]{\linewidth}\raggedright
Journal ID
\end{minipage} & \begin{minipage}[b]{\linewidth}\raggedright
Journal Name
\end{minipage} & \begin{minipage}[b]{\linewidth}\raggedright
First Run Count
\end{minipage} & \begin{minipage}[b]{\linewidth}\raggedright
OpenAlex Total Count
\end{minipage} \\
\midrule\noalign{}
\endhead
\bottomrule\noalign{}
\endlastfoot
S2764628096 & Journal of Agriculture Food Systems and Community
Development & 57 & 825 \\
S115427279 & Public Health Nutrition & 51 & 3282 \\
S206696595 & Journal of Nutrition Education and Behavior & 41 & 3509 \\
S15239247 & International Journal of Environmental Research and Public
Health & 39 & 59130 \\
S4210201861 & Applied Economic Perspectives and Policy & 39 & 647 \\
S10134376 & Sustainability & 35 & 87533 \\
S5832799 & Journal of Soil and Water Conservation & 34 & 556 \\
S2739393555 & Journal of Agricultural and Applied Economics & 34 &
329 \\
S202381698 & PLoS ONE & 30 & 143568 \\
S124372222 & Renewable Agriculture and Food Systems & 30 & 426 \\
S200437886 & BMC Public Health & 28 & 18120 \\
S91754907 & American Journal of Agricultural Economics & 28 & 876 \\
S18733340 & Journal of the Academy of Nutrition and Dietetics & 27 &
5301 \\
S78512408 & Agriculture and Human Values & 27 & 938 \\
S110785341 & Nutrients & 25 & 30911 \\
S2764593300 & Agricultural and Resource Economics Review & 25 & 247 \\
S4210212157 & Frontiers in Sustainable Food Systems & 23 & 3776 \\
S63571384 & Food Policy & 20 & 1069 \\
S69340840 & The Journal of Rural Health & 20 & 749 \\
S4210234824 & EDIS & 18 & 3714 \\
S19383905 & Agricultural Finance Review & 18 & 327 \\
S119228529 & Journal of Hunger \& Environmental Nutrition & 17 & 467 \\
S43295729 & Remote Sensing & 14 & 33899 \\
S2738397068 & Land & 14 & 9774 \\
S80485027 & Land Use Policy & 14 & 4559 \\
\end{longtable}

\begin{longtable}[]{@{}
  >{\raggedright\arraybackslash}p{(\columnwidth - 6\tabcolsep) * \real{0.1818}}
  >{\raggedright\arraybackslash}p{(\columnwidth - 6\tabcolsep) * \real{0.3117}}
  >{\raggedright\arraybackslash}p{(\columnwidth - 6\tabcolsep) * \real{0.2208}}
  >{\raggedright\arraybackslash}p{(\columnwidth - 6\tabcolsep) * \real{0.2857}}@{}}
\caption{(\#tab:top25authors) Top 25 Authors by First Run
Count}\tabularnewline
\toprule\noalign{}
\begin{minipage}[b]{\linewidth}\raggedright
Author ID
\end{minipage} & \begin{minipage}[b]{\linewidth}\raggedright
Author Name
\end{minipage} & \begin{minipage}[b]{\linewidth}\raggedright
First Run Count
\end{minipage} & \begin{minipage}[b]{\linewidth}\raggedright
OpenAlex Total Count
\end{minipage} \\
\midrule\noalign{}
\endfirsthead
\toprule\noalign{}
\begin{minipage}[b]{\linewidth}\raggedright
Author ID
\end{minipage} & \begin{minipage}[b]{\linewidth}\raggedright
Author Name
\end{minipage} & \begin{minipage}[b]{\linewidth}\raggedright
First Run Count
\end{minipage} & \begin{minipage}[b]{\linewidth}\raggedright
OpenAlex Total Count
\end{minipage} \\
\midrule\noalign{}
\endhead
\bottomrule\noalign{}
\endlastfoot
A5016803484 & Heather A. Eicher‐Miller & 15 & 140 \\
A5024975191 & Edward A. Frongillo & 13 & 351 \\
A5055158106 & Becca B.R. Jablonski & 12 & 60 \\
A5047780964 & Meredith T. Niles & 11 & 200 \\
A5076121862 & Sheri D. Weiser & 10 & 241 \\
A5068812455 & Cindy W. Leung & 10 & 170 \\
A5062679478 & J. Gordon Arbuckle & 10 & 68 \\
A5015017711 & Jeffrey K. O'Hara & 10 & 27 \\
A5081656928 & Whitney E. Zahnd & 9 & 147 \\
A5002438645 & Phyllis C. Tien & 8 & 244 \\
A5035584432 & Angela D. Liese & 8 & 172 \\
A5027684365 & Dayton M. Lambert & 8 & 110 \\
A5081012770 & Linda J. Young & 8 & 51 \\
A5008463933 & Catherine Brinkley & 8 & 34 \\
A5030548116 & Michele Ver Ploeg & 8 & 33 \\
A5056021318 & Nathan Hendricks & 7 & 320 \\
A5024248662 & Adebola Adedimeji & 7 & 137 \\
A5002732604 & Julia A. Wolfson & 7 & 137 \\
A5038610136 & Christopher N. Boyer & 7 & 115 \\
A5044317355 & Daniel Merenstein & 7 & 113 \\
A5006129622 & Carmen Byker Shanks & 7 & 103 \\
A5060802257 & Tracey E. Wilson & 7 & 102 \\
A5050792105 & Jennifer L. Moss & 7 & 90 \\
A5032940306 & Lisa Harnack & 7 & 89 \\
A5024127854 & Eduardo Villamor & 7 & 84 \\
\end{longtable}

The resulting seed corpus contained approximately 1,774,245
(\textless{}\textgreater) unique publications. An initial download of
full-texts achieved a success rate of approximately 35\%, corresponding
to an estimated 625,000 accessible full texts. Local full-text searches
were planned using this subset to improve detection of dataset mentions
beyond what was possible through OpenAlex's built-in search functions.

While the seed corpus approach allowed more targeted retrieval,
limitations remain. Full-text download success was constrained by
incomplete or inaccessible open-access links, and processing the entire
corpus was computationally intensive. Future efforts may require
distributed processing or refined selection criteria to further improve
efficiency.

Results from both methods are compared to assess differences in dataset
mention detection across approaches.

\section{\texorpdfstring{\Large Results Comparing Publication and
Journal
Coverage}{Results Comparing Publication and Journal Coverage}}\label{results-comparing-publication-and-journal-coverage}

An objective of this report is to understand differences in publication
coverage across Scopus and OpenAlex. Specifically, we ask: (1) how many
and which publications referencing the Ag Census appear in each citation
database, and (2) how many and which journals publishing these articles
overlap between the two sources. In addition, this analysis aims to
determine whether the different search approaches used to find dataset
mentions in OpenAlex publications---the full-text search versus a
seed-corpus approach---produce substantially different coverage.

This section presents two sets of results. In both cases, a seed-corpus
approach is used to identify Ag Census publications in Scopus. We then
compare these results to (i) publications identified through a full-text
search in OpenAlex and (ii) publications identified using a seed-corpus
approach in OpenAlex.

\subsection{\texorpdfstring{\Large Method 1 (Full Text
Search)}{Method 1 (Full Text Search)}}\label{method-1-full-text-search}

The first comparison looks at publication and journal coverage between
the Scopus (using the seed-corpus method) and OpenAlex (using the
full-text search). The results describe how many publications mention
the Ag Census in each database and evaluates the degree of overlap
between them.

\pagebreak

Table 1 displays the coverage of publications that referenced the Ag
Census data. The table distinguishes between results that appear in both
databases (column 1), only in OpenAlex (column 2), and only in Scopus
(column 3). Column 4 is a total of all distinct publications or journals
across both databases. The first row reports the number of individual
publications found in each category. The unit of analysis is at the
publication level. The second row reports the number of journals that
include the Ag Census publications, again broken out by their appearance
in one or both datasets. The unit of analysis is at the journal level.

\textbf{ADD TABLE 1 HERE}

The main takeaway from Table 1 is that there is little overlap in
publications and journals between the two databases when using
OpenAlex's full-text search. According to this table, there are 5,473
unique publications referencing the Ag Census across both citation
databases, appearing in 2,686 unique journals, identified by their
ISSNs. The number of overlapping publications is 505 (9.23\%), with the
majority of publications referencing Ag Census are picked up only by
Scopus (4,207 or 76.87\%), and a smaller share is identified only in
OpenAlex (761 or 13.9\%).

Journal coverage shows a similar pattern. Of the journals that include
at least one Ag Census publication, 247 (9.2\%) are shared between the
two databases, 2,362 (87.9\%) are found only in Scopus, 77 (2.9\%) only
in OpenAlex.

These results show that the coverage of publications mentioning the Ag
Census and the journals in which they are found is much more extensive
in Scopus, and suggest that OpenAlex's full-text search along may miss
many dataset mentions.

\subsection{\texorpdfstring{\Large Method 2 (Seed Corpus
Approach)}{Method 2 (Seed Corpus Approach)}}\label{method-2-seed-corpus-approach}

Based on the pattern observed from the full-text search in OpenAlex, the
differences likely arise, at least in part, from limitations in how
OpenAlex processes and indexes text. Specifically, we found that
OpenAlex's full-text search does not index references as searchable
text---they are stored as pointers, not included in the searchable body
of the publication. In addition, n-gram-level metadata that might
capture mentions of the Ag Census outside the main text was not
accessible for the full set of publications. To address these
limitations and create a more consistent comparison with Scopus, we
applied a seed-corpus approach to OpenAlex, targeting a curated set of
authors, journals, and topics associated with Ag Census use.

\pagebreak

While this method can overcome the limitations of the OpenAlex full-text
search, it is computationally more intensive. To limit the cost of the
search corpus method, a list of the top authors, topics, and journals is
provided, as described above. This list serves as a set of filters
through which the search corpus is applied. Mentions of the Ag Census
are then searched for within publications meeting these criteria.

\textbf{ADD TABLE 2 HERE}

Next, we compare the degree of overlap across the different search
methods, focusing on OpenAlex publications that also overlap with Scopus
records. Specifically, we examine whether the OpenAlex full-text search
and the OpenAlex seed-corpus approach identified the same publications
and journals referencing the Census of Agriculture, or whether each
method uncovered distinct sets of works.

Table 3 summarizes the overlap between publications identified through
the two OpenAlex search methods, restricted to publications that also
appear in Scopus. As reported earlier, 505 publications were identified
in both Scopus and OpenAlex using the OpenAlex full-text search (Table
1), while 363 publications were identified using the OpenAlex
seed-corpus approach (Table 2). Table 3 specifically shows how many of
the 505 full-text search publications were also found in the seed-corpus
set. The comparison reveals that 105 publications are shared between the
two methods, representing 20.8\% of the full-text set and 28.9\% of the
seed-corpus set. These results suggest that the choice of search
strategy meaningfully influences which Scopus-linked publications are
recovered in OpenAlex.

\textbf{ADD TABLE 3 HERE}

Table 4 summarizes the overlap between journals identified through the
two OpenAlex search methods, restricted to journals linked to
publications that also appear in Scopus. As noted previously, the
OpenAlex full-text search and seed-corpus approach each identified a set
of journals containing publications referencing the Census of
Agriculture. Table 4 specifically shows how many journals were common to
both sets. A total of 137 journals were shared between the two methods,
representing 55.5\% of the full-text set and 49.6\% of the seed-corpus
set. Among these shared journals, 19 were part of the original list of
top journals used to construct the seed corpus.

\textbf{ADD TABLE 4 HERE}

\section{\texorpdfstring{\Large Results Comparing Publication Coverage
Disaggregated by
Journal}{Results Comparing Publication Coverage Disaggregated by Journal}}\label{results-comparing-publication-coverage-disaggregated-by-journal}

Now that we have compared journal coverage across the two citation
databases, we next examine the publications within journals that are
indexed in both Scopus and OpenAlex. We report these results for the
full-text search approach and the seed-corpus approach in OpenAlex.

\subsection{\texorpdfstring{\Large Method 1 (Full-Text
Search)}{Method 1 (Full-Text Search)}}\label{method-1-full-text-search-1}

Table 5 provides journal-level detail on the overlap between
publications indexed in Scopus and OpenAlex, based on the full-text
search results in OpenAlex. Each row corresponds to a journal included
in the analysis. The set of journals included here matches the group of
overlapping journals reported in Table 1---that is, journals where
publications were found in both Scopus and OpenAlex.

The table is divided into two sections: overlap statistics and
publication counts. The overlap statistics report three measures. The
percentage labeled ``\% Both'' indicates the share of a journal's
publications that were found in both Scopus and OpenAlex. ``\% Scopus''
shows the share of publications that appeared only in Scopus, while ``\%
OpenAlex'' shows the share of publications that appeared only in
OpenAlex. Together, these columns summarize how consistently each
journal's publications are represented across the two databases.

The publication counts section reports the number of overlapping and
non-overlapping publications for each journal. ``Pubs Both'' shows the
number of publications found in both Scopus and OpenAlex, while ``Pubs
Scopus'' and ``Pubs OpenAlex'' show the number of publications found
exclusively in one database. The final column, ``Total Pubs,'' provides
the total number of distinct publications associated with each journal
across both databases.

Finally, the table also includes each journal's 2022 Scopus CiteScore to
provide additional context on journal prominence.

Reading across a row, the table allows for direct comparison of database
coverage at the journal level, highlighting journals where coverage
between Scopus and OpenAlex aligns closely and those where substantial
gaps exist.

\textbf{ADD TABLE 5 HERE}

To further characterize database coverage, we calculated weighted
average CiteScores for the journals analyzed in Table 5. The weighted
averages account for both the number of publications associated with
each journal and the journal's 2022 Scopus CiteScore. For Scopus, the
weighted average CiteScore, calculated by weighting each journal's
CiteScore by its number of publications found in Scopus (including
shared publications), is approximately 12.99. For OpenAlex, the weighted
average CiteScore, using the number of publications found in OpenAlex
(including shared publications), is slightly higher at 13.23.

These results suggest that, among the overlapping journals, OpenAlex
tends to recover a slightly more citation-intensive set of publications
compared to Scopus. However, the difference in weighted averages is
modest, indicating broadly similar coverage of higher-impact journals
across the two databases. To formally test the difference in weighted
average CiteScores between Scopus and OpenAlex, we calculated a
t-statistic for the difference in means. The resulting t-statistic is
approximately -0.15, indicating that the difference between the two
weighted averages is not statistically significant. This suggests that,
within the set of overlapping journals analyzed, Scopus and OpenAlex
recover a similarly citation-intensive set of publications.

\subsection{\texorpdfstring{\Large Method 2 (Seed Corpus
Approach)}{Method 2 (Seed Corpus Approach)}}\label{method-2-seed-corpus-approach-1}

We next examine the results obtained using the OpenAlex seed corpus
approach.

Table 6 summarizes publication-level overlap statistics for journals
identified using the seed corpus approach. As with Table 5, the table is
restricted to journals associated with publications that overlap between
Scopus and OpenAlex. The overlap statistics show the percentage of each
journal's publications found in both databases (\% Both), found only in
Scopus (\% Scopus), or found only in OpenAlex (\% OpenAlex). Publication
counts are also reported separately for shared and database-specific
publications, along with the total number of publications associated
with each journal. Each journal's 2022 Scopus CiteScore is included to
provide additional context on journal prominence.

\textbf{ADD TABLE 6 HERE}

To further compare database coverage under the seed corpus approach, we
calculated weighted average CiteScores for Scopus and OpenAlex. The
weighted averages account for both the number of publications associated
with each journal and the journal's 2022 Scopus CiteScore. For Scopus,
the weighted average CiteScore is approximately 11.90, while for
OpenAlex it is higher at approximately 14.10.

A t-statistic for the difference in means was also calculated to assess
whether the difference is statistically meaningful. The resulting
t-statistic is approximately -2.18, indicating that the difference is
statistically significant. These results suggest that, when using the
seed corpus approach, OpenAlex recovers a more citation-intensive set of
publications compared to Scopus.

\section{\texorpdfstring{\Large Summary of
Findings}{Summary of Findings}}\label{summary-of-findings}

This report compares the coverage of publications and journals
referencing the Census of Agriculture across Scopus and OpenAlex, using
two approaches for identifying relevant OpenAlex publications: a
full-text search and a seed corpus approach.

Using the full-text search in OpenAlex, we found relatively limited
overlap with Scopus. Only 9.2\% of publications and 9.2\% of journals
referencing the Census of Agriculture appeared in both databases, with
Scopus identifying a substantially larger share of relevant works. These
results suggest that relying solely on OpenAlex's full-text search may
miss a significant number of dataset mentions.

Applying the seed corpus approach to OpenAlex improved overlap with
Scopus and provided a more structured way to capture publications
associated with known journals, authors, and topics. However, the
percentage of overlapping publications referencing the Census of
Agriculture is lower at 6.42\% even though there is a slightly higher
percentage of shared journals at 10.73\%.

Comparing the overlap between the two OpenAlex methods reveals
differences in underlying samples. Only 20.8\% of full-text search
publications were also found in the seed corpus set, and 28.9\% of seed
corpus publications matched those found in the full-text search.
Journal-level overlap was somewhat higher, with 137 journals shared
between the two methods (representing approximately 50--55\% overlap
across the two pools).

Publication-level comparisons disaggregated by journal showed that,
under the full-text search approach, OpenAlex and Scopus recovered
broadly similar sets of citation-intensive publications. The weighted
average CiteScores were 13.23 for OpenAlex and 12.99 for Scopus, with a
t-statistic of -0.15, indicating no statistically significant
difference. In contrast, under the seed corpus approach, OpenAlex
recovered a more citation-intensive set of publications, with a weighted
average CiteScore of 14.10 compared to 11.90 for Scopus. The t-statistic
of -2.18 indicates that this difference is statistically significant.

It is important to note that the full-text search and seed corpus
approaches represent two distinct sampling methods within OpenAlex. The
full-text search attempts to identify dataset mentions directly from the
body of text available for a subset of publications, while the seed
corpus approach relies on pre-selected journals, topics, and authors
more likely to reference the Census of Agriculture. As a result, the
pools of publications identified by each method are not strictly
comparable: they are drawn from different underlying subsets of
OpenAlex's catalog. This context is important for interpreting
differences in coverage and citation intensity across the two
approaches.

\subsection*{Institutional Comparison}\label{institutional-comparison}
\addcontentsline{toc}{subsection}{Institutional Comparison}

The results in this section compare institutional coverage across three
major bibliographic databases: Scopus, OpenAlex, and Dimensions.

Each of the featured citation databases represent some portion of the
global research landscape, yet their inclusion criteria and
institutional coverage vary in ways that may inform disparities. Our
goal is to assess which institutions are represented in each source,
with particular attention to coverage of underrepresented and
Minority-Serving Institutions (MSIs). By building a harmonized dataset
that links IPEDS identifiers to institutional records across Scopus,
OpenAlex, and Dimensions, we aim to evaluate how these citation
databases include or exclude institutions across institutional
characteristics such as control, level, MSI status, and geography. This
analysis informs a broader conversation about equity, transparency, and
accountability in research data systems.

\subsubsection*{Step 1: Clean IPEDS and MSI
data}\label{step-1-clean-ipeds-and-msi-data}
\addcontentsline{toc}{subsubsection}{Step 1: Clean IPEDS and MSI data}

In this section, I document the construction and visualization of MSI
(Minority-Serving Institution) eligibility trends from 2017 to 2023, as
part of the broader effort to compare institutional coverage across
Scopus, OpenAlex, and Dimensions. To support that analysis, we needed a
clean, panel-form dataset of U.S. higher education institutions,
including consistent MSI designations over time. I compiled and cleaned
data from multiple sources---the MSI Data Project (Nguyen et al., 2023)
for 2017--2021 and Rutgers CMSI lists for 2022--2023---and merged these
with IPEDS institutional data, filtered to include only 2- and 4-year
institutions in the 50 U.S. states. I addressed inconsistencies in
eligibility labels, resolved duplicates, and created summary measures of
MSI eligibility by year. The resulting visualization highlights both the
number and percent of institutions designated as MSIs over time, with a
sharp increase observed in 2022. The accompanying plot and source code
are included for transparency. All additional details are available in
the IPEDS\footnote{\href{appendices/app_ipeds.qmd}{IPEDS appendix
  available here}} and MSI\footnote{\href{appendices/app_msi.qmd}{MSI
  appendix available here}} Appendices.

Source code used to generate graphic: Available here.




\end{document}
